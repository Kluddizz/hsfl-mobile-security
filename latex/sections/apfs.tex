\section{APFS}
Das Apple File System (APFS) ist das seit iOS 10.3 auf Mobilgeraeten und seit macOS High Sierra 10.13 auf Computern verwendete Dateisystem von Apple. Es ist der Nachfolger von HFS+ und aehnelt den Dateisystemen BRTFS und ZFS in einigen Punkten, hat jedoch vor allem im Bereich der Sicherheit und Verschluesselung einige zusaetzliche Features.

\subsection{Volumes und Container}
Unter APFS ist es moeglich mehrere Volumes auf demselben physischen Speicher zu verwenden. Ein Volume befindet sich dabei immer in einem Container. Wird also ein Speichermedium mit einer Speicherkapazitaet von 500GiB verwendet, ist es also moeglich einen zum Beispiel 400GiB grossen APFS-Container zu erstellen. In diesem Container koennen dann beliebig viele Volumes mit einer Groesse von 400GiB erstellt werden.\\
Wird dies getan, so benutzen alle Volumes dieselben 400GiB des Speichermediums, konkurrieren andererseits damit aber auch um denselben Speicherplatz. Werden zum Beispiel 3 400GiB grosse Volumes auf dem Container erstellt, wobei die einzelnen Volumes jeweils tatsaechlich 100GiB, 30GiB und 20GiB verwenden, so ist der freie Platz eines jeden Volumes 250GiB.\\
Sollte dabei der Speicherplatz des Containers nicht mehr ausreichen, so ist es moeglich den Container zu erweitern. Auf diese Weise werden logische Volumes von physischen Speichergegebenheiten abstrahiert und ermoeglichen eine variablere und dynamischere Nutzung des verfuegbaren Speichers.
\subsection{Datensicherheit}

\subsubsection{Daten}
Zentraler Punkt der Verschluesselung unter iOS mit APFS ist eine von Apple als “Datensicherheit” betitelte Technologie. Diese beeinhaltet vorallem zwei wesentliche Merkmale.\\
Das erste ist, dass jede Datei auf dem Dateisystem mit einem eigenem Schluessel verschluesselt wird. Dies gilt auch, wenn eine Datei geklont wird.\\
Das zweite Merkmal sind die Klassenschluessel, wobei jede Datei einer spezifischen Klasse angehoert. Mit diesem Schluessel wird jede Datei, deren Metadaten sie der entsprechenden Klasse zuordnen nochmals verschluesselt.\\
Auf diese Weise wird gewaehrleistet, dass nur Apps, welche die Zugriffsrechte fuer die entsprechende Klasse besitzen auch wirklich auf die Dateien dieser Klasse zugreifen koennen.

\subsubsection{Metadaten}
Im Gegensatz zu den eigentlichen Dateien, teilen sich alle Metadaten einen Schluessel. Dieser wird einmalig bei der Einrichtung des iOS-Geraetes erstellt und in der “Secure Enclave” gespeichert. Weitergehend wird dieser Schluessel mit einem weiteren Schluesel verschluesselt, der allerdings im “Effaceable Storage” gespeichert ist.\\
Zweck dieses Schluessels ist es ueber die Einstellungen oder per Fernzugriff schnell geloescht werden zu koennen. Auf diese Weise geht der eigentliche Schluessel fuer die Metadaten verloren und saemtliche Daten auf dem Geraet sind unwiderruflich verschluesselt und somit vor dem Zugriff Dritter geschuetzt.
\subsubsection{Hardware}
Die Verschluesselung sowohl der eigentlichen Daten, als auch der Metadaten wird weitergehend dadurch unterstuetzt, dass die eigentlichen Schluessel nie an den Hauptprozessor weitergegeben werden. Stattdessen werden alle Verschluesselungen und Entschluesselungen nur in der Secure Enclave durchgefuehrt.
\subsubsection{Appzugriff}
Der Zugriff von Apps auf Daten des Nutzers wird vor allem durch einen sogenannten Data Vault geregelt. Dieser verwaltet die Zugriffsrechte auf verschiedene Arten von nutzerbezogenen Informationen. Dabei kann der Nutzer jederzeit einsehen, welche Apps welche dieser Zugriffsrechte haben und diese entziehen oder neue gewaehren.\\
Einzige Ausnahme bilden dabei einige Systemapplikationen wie Kalender, Kamera, Medien oder aehnliches.\\
Auf diese Weise hat der Nutzer zu jedem Zeitpunkt volle Information und Kontrolle ueber den Verbleib seiner Daten.

\subsection{Copy-On-Write}
Um unnoetige Kopien oder korrupte Dateien zu vermeiden, also die Intigritaet der gespeicherten Daten sicher zu stellen, setzt APFS auf einen Copy-On-Write Ansatz fuer die Metadaten von Dateien. Diese Metadaten werden also erst geupdated, sobald eine Datei beziehungsweise ein Block des Speichermediums wirklich veraendert werden soll.\\
Durch die Realisierung dieser Veraenderung mittels einer atomaren Transaktion “Atomic Safe-Save” wird das Dateisystem weitergehend vor korrupten Daten geschuetzt.

\subsection{Klonen}
Der Copy-On-Write Ansatz von APFS wird allerdings nicht nur bei den Metadaten eingesetzt, sondern auch bei dem tatsaechlichen Inhalt von Dateien. Wird eine Datei unter APFS geklont, so wird diese nicht komplett physisch auf eine andere Stelle des Speichermediums kopiert. Stattdessen werden lediglich Verweise auf die urspruengliche Datei erstellt. Werden zu dem Klon weitere Bloecke hinzugefuegt oder veraendert, so werden fuer diese Neuerungen weitere Verweise erstellt, waehrend die bestehenden Verweise auf die Ur-Datei erhalten bleiben.\\
Dieser Ansatz beschleunigt das Kopieren von Dateien enorm, da nicht die gesamte Datei kopiert wird, sondern lediglich ein Link - beziehungsweise eine neue Version der Datei - erzeugt wird. Da somit dieselbe Datei nicht zweimal vorhanden sein muss, wird damit ausserdem Speicherplatz gespart.\\
Ein weiterer Vorteil dieses Vorgehens ist es weiterhin, dass es damit einfach ist eine Art Versionshistorie von Dateien zu erstellen und damit gemachte Aenderungen besser nachvollziehen zu koennen oder einfacherer zu vorherigen Versionen zurueck zu springen.\\
Die Nachvollziehbarkeit von Transaktionen wird ausserdem durch eine bessere Aufloesung der Zeitstempel im System (1ns) verglichen mit HFS+ (1s) gefoerdert.

\subsection{Backups}
Das APFS unterstuetzt mit Snapshotes eine Funktion zum Erstellen von Backups des Dateisystems. Dieses funktioniert aehnlich wie das Klonen von Dateien, allerdings mit der Einschraenkung, dass ein Snapshot nicht veraenderbar, also read only ist.\\
Anders formuliert, erstellt ein Snapshot also einen Klon aller Dateien des Dateisystems zum aktuellen Zeitpunkt. Soll das System also mittels Snapshot auf einen vorherigen Zustand zurueckgesetzt werden, so muessen keine Dateien kopiert werden, sondern lediglich die Verweise auf bestehende Dateien veraendert, beziehungsweise auf den Stand des Snapshots zurueckgesetzt werden.\\
Die einzige Ausnahme bilden dabei Bloecke, die veraendert oder geloescht wurden. Diese werden dem Snapshot hinzugefuegt und muessen im Falle einer Wiederherstellung tatsaechlich kopiert werden.

\subsection{Datenbackups mit Verschluesselung}
???
\subsection{Vergleich zu MacOS}
Das APFS wird mittlerweile sowohl auf iOS als auch auf MacOS Geraeten eingesetzt. Allerdings sind die Features von APFS auf iOS und MacOS jedoch unterschiedlich. Der Hauptunterschied ist dabei die Verwendung von FileVault unter MacOS, welcher unter iOS nicht nutzbar ist.\\
FileVault ist dabei dafuer zustaendig, die Volumes des APFS zu verschluesseln. Dieser Schluessel muss bei jedem Startvorgang vom Nutzer unter Eingabe seiner Anmeldeinformationen freigegeben werden.\\
Ein weiterer Unterschied zwischen der Verwendung von APFS unter iOS und MacOS sind ausserdem die benoetigten Volumes eines Containers.
Waehrend iOS mit einem Systemvolume zur Speicherung der Systemdaten und einem Datenvolume fuer die Nutzerdaten auskommt, benoetigt MacOS 3 weitere Volumes. Diese sind zum einen das Preboot-Volume, welches Informationen ueber alle Volumes des Containers enthaelt und vorallem zum Booten benoetigt wird. Zum anderen das VM-Volume, welches dem Austausch von Dateispeicher dient. Und zum letzten dem Wiederherstellungsvolume, welche recoveryOS beeinhaltet und damit fuer Systemwiederherstellungen verantwortlich ist.