\section{Apple File System}
Das \textit{Apple File System} (\textit{APFS}) ist das seit iOS 10.3 auf
Mobilgeräten und seit macOS High Sierra 10.13 auf Computern verwendete
Dateisystem von Apple. Es ist der Nachfolger von \textit{HFS+} und ähnelt den
Dateisystemen BRTFS und ZFS in einigen Punkten, hat jedoch vor allem im Bereich
der Sicherheit und Verschlüsselung einige zusätzliche Features \cite{golem,
apple_filesystem_reference}.

\subsection{Volumes und Container}
Unter \textit{APFS} ist es möglich mehrere Volumes auf demselben physischen
Speicher zu verwenden \cite{about_apfs}. Ein Volume befindet sich dabei immer in
einem Container. Wird also ein Speichermedium mit einer Speicherkapazität von
500 GiB verwendet, ist es also möglich einen zum Beispiel 400 GiB grossen
\textit{APFS}-Container zu erstellen. In diesem Container können dann beliebig
viele Volumes mit einer Größe von 400 GiB erstellt werden.

Wird dies getan, so benutzen alle Volumes dieselben 400 GiB des Speichermediums,
konkurrieren andererseits damit aber auch um denselben Speicherplatz. Werden zum
Beispiel 3 400 GiB grosse Volumes auf dem Container erstellt, wobei die
einzelnen Volumes jeweils tatsächlich 100 GiB, 30 GiB und 20 GiB verwenden, so
ist der freie Platz eines jeden Volumes 250 GiB \cite{golem}.

Sollte dabei der Speicherplatz des Containers nicht mehr ausreichen, so ist es
möglich den Container zu erweitern.  Auf diese Weise werden logische Volumes von
physischen Speichergegebenheiten abstrahiert und ermöglichen eine variable und
dynamische Nutzung des verfügbaren Speichers.

\subsection{Datensicherheit}

\subsubsection{Daten}
Zentraler Punkt der Verschlüsselung unter iOS mit \textit{APFS} ist eine von
Apple als \textit{Datensicherheit} betitelte Technologie \cite[S. 48
ff]{apple2020}. Diese beinhaltet vor allem zwei wesentliche Merkmale.

Das erste ist, dass jede Datei auf dem Dateisystem mit einem eigenem Schlüssel
verschlüsselt wird. Dies gilt auch, wenn eine Datei geklont wird \cite[S.
49]{apple2020}.

Das zweite Merkmal sind die \textit{Klassenschlüssel}, wobei jede Datei einer
spezifischen Klasse angehört. Mit diesem Schlüssel wird jede Datei, deren
Metadaten sie der entsprechenden Klasse zuordnen nochmals verschlüsselt \cite[S.
49]{apple2020}.

Auf diese Weise wird gewährleistet, dass nur Apps, welche die Zugriffsrechte für
die entsprechende Klasse besitzen auch wirklich auf die Dateien dieser Klasse
zugreifen können.

\subsubsection{Metadaten}
Im Gegensatz zu den eigentlichen Dateien, teilen sich alle Metadaten einen
Schluessel \cite[S. 50]{apple2020}. Dieser wird einmalig bei der Einrichtung des
iOS-Gerätes erstellt und in der \textit{Secure Enclave} gespeichert.
Weitergehend wird dieser Schlüssel mit einem weiteren Schlüssel verschlüsselt,
der allerdings im \textit{Effaceable Storage} gespeichert ist.

Zweck dieses Schlüssels ist es über die Einstellungen oder per Fernzugriff
schnell gelöscht werden zu können. Auf diese Weise geht der eigentliche
Schlüssel für die Metadaten verloren und sämtliche Daten auf dem Gerät sind
unwiderruflich verschlüsselt und somit vor dem Zugriff Dritter geschützt.

\subsubsection{Hardware}
Die Verschlüsselung sowohl der eigentlichen Daten, als auch der Metadaten wird
weitergehend dadurch unterstützt, dass die eigentlichen Schlüssel nie an den
Hauptprozessor weitergegeben werden \cite[S. 50]{apple2020}. Stattdessen werden
alle Verschlüsselungen und Entschlüsselungen nur in der \textit{Secure Enclave}
durchgeführt.

\subsubsection{Appzugriff}
Der Zugriff von Apps auf Daten des Nutzers wird vor allem durch einen
sogenannten \textit{Data Vault} geregelt \cite[S. 47]{apple2020}. Dieser
verwaltet die Zugriffsrechte auf verschiedene Arten von nutzerbezogenen
Informationen. Dabei kann der Nutzer jederzeit einsehen, welche Apps welche
dieser Zugriffsrechte haben und diese entziehen oder neue gewähren.

Einzige Ausnahme bilden dabei einige Systemapplikationen wie Kalender, Kamera,
Medien oder aehnliches. Diese Systemapplikationen erfordern auf Grund ihrer
Funktionalität einen Zugriff auf die persönlichen Daten und setzen diesen
deshalb voraus.

Auf diese Weise hat der Nutzer zu jedem Zeitpunkt volle Information und
Kontrolle ueber den Verbleib seiner Daten.

\subsection{Copy-On-Write}
Um unnötige Kopien oder korrupte Dateien zu vermeiden, also die Integrität der
gespeicherten Daten sicher zu stellen, setzt \textit{APFS} auf einen
Copy-On-Write Ansatz fuer die Metadaten von Dateien \cite{golem}. Diese Metadaten
werden also erst geupdatet, sobald eine Datei beziehungsweise ein Block des
Speichermediums wirklich verändert werden soll.

Durch die Realisierung dieser Veränderung mittels einer atomaren Transaktion
\textit{Atomic Safe-Save} wird das Dateisystem weitergehend vor korrupten Daten
geschützt, da auf diese Weise sichergestellt wird, dass eine Transaktion
entweder vollständig oder gar nicht durchgeführt wird.


\subsection{Klonen}
Der Copy-On-Write Ansatz vom \textit{APFS} wird allerdings nicht nur bei den
Metadaten eingesetzt, sondern auch bei dem tatsächlichen Inhalt von Dateien.
Wird eine Datei unter APFS geklont, so wird diese nicht komplett physisch auf
eine andere Stelle des Speichermediums kopiert. Stattdessen werden lediglich
Verweise auf die ursprüngliche Datei erstellt. Werden zu dem Klon weitere Blöcke
hinzugefügt oder verändert, so werden für diese Neuerungen weitere Verweise
erstellt, während die bestehenden Verweise auf die Ur-Datei erhalten bleiben.

Dieser Ansatz beschleunigt das Kopieren von Dateien enorm, da nicht die gesamte
Datei kopiert wird, sondern lediglich ein Link -- beziehungsweise eine neue
Version der Datei -- erzeugt wird \cite{about_apfs}. Da somit dieselbe Datei nicht
zweimal vorhanden sein muss, wird damit außerdem Speicherplatz gespart.

Ein weiterer Vorteil dieses Vorgehens ist es weiterhin, dass es damit einfach
ist eine Art Versionshistorie von Dateien zu erstellen und damit gemachte
Änderungen besser nachvollziehen zu können oder einfacherer zu vorherigen
Versionen zurück zu springen.

Die Nachvollziehbarkeit von Transaktionen wird außerdem durch eine bessere
Auflösung der Zeitstempel im System (1 ns) verglichen mit \textit{HFS+} (1 s)
gefördert \cite{arstechnica}.


\subsection{Backups}
Das \textit{APFS} unterstützt mit \textit{Snapshots} eine Funktion zum Erstellen
von Backups des Dateisystems. Dieses funktioniert ähnlich wie das Klonen von
Dateien, allerdings mit der Einschränkung, dass ein Snapshot nicht veränderbar,
also read only ist \cite{arstechnica}.

Anders formuliert, erstellt ein Snapshot also einen Klon aller Dateien des
Dateisystems zum aktuellen Zeitpunkt. Soll das System also mittels Snapshot auf
einen vorherigen Zustand zurückgesetzt werden, so müssen keine Dateien kopiert
werden, sondern lediglich die Verweise auf bestehende Dateien verändert,
beziehungsweise auf den Stand des Snapshots zurückgesetzt werden.

Die einzige Ausnahme bilden dabei Bloecke, die verändert
oder gelöscht wurden. Diese werden dem Snapshot hinzugefügt und müssen im Falle
einer Wiederherstellung tatsächlich kopiert werden.

Durch die standardmäßige Verschlüsselung von Daten mittels Klassen- und
Dateischluesseln, welche beim klonen von Dateien bestehen bleibt sind Backups
automatisch verschlüsselt \cite[S. 50]{apple2020}. Da die UID des Gerätes Teil
der Klassenschlüssel ist \cite[S. 50]{apple2020}, wird außerdem gewährleistet,
dass Backups nur auf dem Gerät auf dem sie erstellt wurden wiederhergestellt
werden können. Bei Verlust des Speichermediums ist ein Auslesen des Snapshots
auf anderen Geräten also nicht möglich.


\subsection{Vergleich zu MacOS}
Das \textit{APFS} wird mittlerweile sowohl auf iOS als auch auf MacOS Geräten
eingesetzt. Allerdings sind die Features von APFS auf iOS und MacOS jedoch
unterschiedlich. Der Hauptunterschied ist dabei die Verwendung von
\textit{FileVault} unter MacOS \cite[S. 56 ff]{apple2020}, im Gegensatz zur
\textit{Datensicherheit} unter iOS \cite[S. 48 ff]{apple2020}.

\textit{FileVault} ist dabei dafuer zustaendig, die Volumes des \textit{APFS} zu
verschlüsseln. Dieser Schlüssel muss bei jedem Startvorgang vom Nutzer unter
Eingabe seiner Anmeldeinformationen freigegeben werden.

Ein weiterer Unterschied zwischen der Verwendung vom \textit{APFS} unter iOS und
MacOS sind außerdem die benötigten Volumes eines Containers.  Während iOS mit
einem Systemvolume zur Speicherung der Systemdaten und einem Datenvolume fuer
die Nutzerdaten auskommt \cite[S. 49]{apple2020}, benötigt MacOS 3 weitere
Volumes \cite[S. 48]{apple2020}. Diese sind zum einen das Preboot-Volume,
welches Informationen über alle Volumes des Containers enthält und vor allem zum
Booten benötigt wird. Zum anderen das VM-Volume, welches dem Austausch von
Datenspeicher dient. Und zum letzten dem Wiederherstellungsvolume, welches
\textit{recoveryOS} beinhaltet und damit für Systemwiederherstellungen
verantwortlich ist.
