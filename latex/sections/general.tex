\section{Sicherheit im Alltag}
Für die Sicherheit im Alltag hat Apple eine Vielzahl von Systemmechanismen implementiert, 
die den Schutz der Benutzerdaten sicherstellen. Die Dateiverschlüsselung bei Endgeräten mit iOS- und 
iPadOS, als Betriebssysteme, wird von Apple als Datensicherheit bezeichnet und ist das Pendant zum 
FileVault unter macOS \cite{apple2020}.

Der Ausgangspunkt für die Hierarchien der Schlüsselverwaltung befindet sich bei Geräten 
mit SEP (Secure Enclave Processor, ab A7 und neuer) im dedizierten Silizium der Secure Enclave. 
Hier werden die Schlüssel und biometrische Daten für die Verwendung von Touch ID und Face ID 
hinterlegt. Dieser isolierte Bereich ist für den AP (Application Processor) nicht zugänglich, wenngleich 
das RAM von beiden Prozessoren geteilt wird, denn der Speicherbereich des SEP
ist verschlüsselt \cite{apple2020}.

Als weiteren Sicherheitsmechanismus werden Zugriffssteuerungen im Kernel erzwungen. Dadurch wird der unbefugte 
Zugriff auf Daten verhindert. Der Datenzugriff von Apps wird in der Regel durch Sandboxing realisiert und durch die 
Forcierung von Data Vaults ergänzt. Simplifiziert sind Data Vaults invertierte Sandboxes, die nicht die Aufrufe einer App, 
sondern den Zugriff auf geschützte Daten einschränken \cite{apple2020}.

Die offensichtlichen Sicherheitsmechanismen, die den Benutzer im All\-tag schü\-tzen, werden im Folgenden vorgestellt.

\subsection{Gerätecodes}
Ein Gerätecode wird von dem Benutzer des Geräts eingegeben, um sicherzustellen,
dass dieser die erforderlichen Zugriffsrechte besitzt. Ist ein Gerätecode
eingerichtet, so wird gleichzeitig automatisch die Datensicherheit aktiviert.
Das bedeutet, dass die Dateien des Systems verschlüsselt vorliegen und je nach
Datensicherheitsklasse beim Entsperren des Geräts entschlüsselt werden. Ein
Gerätecode besteht aus vier oder sechs Ziffern, jedoch können auch
alphanumerische Zeichenfolgen (\texttt{a-z, A-Z, 0-9}) mit beliebiger Länge als
Gerätecode eingerichtet werden. Natürlich haben diese Codes dann eine hö\-he\-re
Sicherheit als die Varianten, die nur aus Ziffern bestehen. Wichtig zu erwähnen
ist außerdem, dass die Gerätecodes mit der UID des Geräts verbunden sind. Eine
Authentifizierung ist deshalb nur auf dem Zielgerät möglich und stellt eine
Sicherheitsvorkehrung dar, da das Gerät im physischen Besitz des Angreifer sein
muss. Selbst wenn das der Fall ist, existiert ein Mechanismus, der den
Authentifizierungsprozess auf 80 Millisekunden festsetzt. Damit sind
Brute-Force-Angriffe schon bei vierstelligen Gerätecodes, bestehend aus Ziffern,
sehr unattraktiv für Angreifer. Zusätzlich können Touch ID oder Face ID als
Gerätecodes verwendet werden. Dies bietet dem Benutzer eine bessere
User-Experience und eine stärkere Sicherheit beim Verschlüsseln von Daten
\cite{apple2020}.

\subsection{Sicherheit vor unautorisierten  Datenverbindungen}
Physisch angeschlossene Geräte,  z.B. Mac, PC oder Zubehör, werden über die Lightning-, USB- oder Smart 
Connector-Schnittstelle mit dem Apple-Gerät verbunden. Aufgrund der Tatsache, dass die Ökosysteme der
angeschlossenen Geräte vielfältig sein können und es keine kryptographisch zuverlässige Möglichkeit gibt,
die Geräte vor einer Datenverbindung zu identifizieren, verlangen iPhones bzw. iPads die Eingabe des Gerätecodes,
bevor eine Datenverbindung aufgebaut werden kann \cite[S. 65]{apple2020}. Dies verhindert den unautorisierten
Zugriff auf die Daten durch ein Fremdsystem. Nachdem die Datenverbindung zu einem Fremdsystem autorisiert wurde, 
muss der Gerätecode für eine bestimmte Zeitspanne nicht mehr für den Verbindungsaufbau zu dem System 
eingegeben werden. Dadurch können Nutzer, die regelmäßige Verbindungen zu bestimmten Geräten herstellen möchten,
für eine Zeitspanne auf eine erneute Authentifizierung verzichten
\cite{apple2020}.

\subsection{Schutz persönlicher Daten}
Mobile Anwendungen benötigen oftmals den Zugriff auf bestimmte Funktionaliten oder Daten des Smartphones.
Diese stehen den Anwendungen aber nicht einfach zur Verfügung. Nutzer können für jede App festlegen, auf 
welche personenbezogene Informationen zugegriffen werden darf. Realisiert wird dies mit verschiedenen Technologien 
wie bspw. dem Data Vault. In den Einstellungen des Betriebssystems kann der Nutzer einer App die Rechte für den Zugriff 
auf individuelle Informationen geben. In einigen systemrelevanten Apps von iOS und iPadOS wird der Zugriff jedoch erzwungen, 
dazu zählen bspw. Kalender, Kamera, Bluetooth und Fitness (Health)
\cite{apple2020}.

Seit der iOS bzw. iPadOS Version 13.4 werden die Daten der Apps von externen Anbieter (nicht Apple) in einem Data Vault geschützt, 
um den Schutz vor unbefugten Datenzugriffen zu gewährleisten. Die Apps unter iOS und iPadOS erhalten bei der Nutzung von iCloud 
standardmäßig Zugriff auf die iCloud Drive. Der Nutzer kann den iCloud-Zugriff
jedoch in den Einstellungen ändern und einschränken \cite{apple2020}.

\subsection{Klassifizierung der Sicherheitschips}
Im Rahmen dieser Ausarbeitung werden oftmals die Sicherheitschips (SoCs - System on Chip) der mobilen Endgeräte thematisiert. 
Die meisten dieser Sicherheitschips basieren auf einer ARM-Architektur und werden in die meisten der Apple-Produkte wie bspw.
iPhone, iPad, Apple Watch, etc. verbaut. Die Chips unterscheiden sich dabei u.a. durch unterschiedliche Anwendungszwecke, 
Fertigungstechnologien, integrierte Co-Prozessoren, AI-Be\-schleu\-ni\-ger und der Anzahl der integrierten Transistoren. Für eine 
bessere Übersicht werden folgend die verschiedenen SoC-Serien, und die Produkte
in denen sie verbaut werden, aufgelistet \cite{apple2020}.

\begin{itemize}
\item \textit{A series}: iPhone und iPad
\item \textit{S series}: Apple Watch
\item \textit{T series} series: macOS-Produkte (MacBook, Mac-Computer)
\item \textit{W series}: AirPods und HomePod
\end{itemize}



