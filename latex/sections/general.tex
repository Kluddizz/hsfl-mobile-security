\section{Sicherheit im Alltag}
Für die Sicherheit im Alltag hat Apple eine Vielzahl von Systemmechanismen implementiert, 
die den Schutz der Benutzerdaten sicherstellen. Die Dateiverschlüsselung bei Endgeräten mit iOS- und 
iPadOS als Betriebssysteme wird von Apple als Datensicherheit bezeichnet und ist das Pendant zum 
FileVault unter macOS. \\

Der Ausgangspunkt für die Hierarchien der Schlüsselverwaltung befindet sich bei Geräten 
mit SEP (Secure Enclave Processor, ab A7 und neuer) im dedizierten Silizium der Secure Enclave. 
Hier werden die Schlüssel und biometrische Daten für die Verwendung von Touch ID und Face ID 
hinterlegt. Dieser isolierte Bereich ist für den AP (Application Processor) nicht zugänglich, wenngleich 
das RAM von beiden Prozessoren geteilt wird, denn der Speicherbereich des SEP ist verschlüsselt. \\

Als weiteren Sicherheitsmechanismus werden Zugriffssteuerungen im Kernel erzwungen. Dadurch wird der unbefugte 
Zugriff auf Daten verhindert. Der Datenzugriff von Apps wird in der Regel durch Sandboxing realisiert und durch die 
Forcierung von Data Vaults ergänzt. Simplifiziert sind Data Vaults invertierte Sandboxes, die nicht die Aufrufe einer App, 
sondern den Zugriff auf geschützte Daten einschränken. \\

Die offensichtlichen Sicherheitsmechanismen, die der Benutzer im Alltag erfährt, werden im Folgenden vorgestellt.

\subsection{Gerätecodes}
Mit der Einrichtung eines Gerätecodes wird automatisch die Datensicherheit, also die Dateiverschlüsselung, aktiviert. 
Die Gerätecodes werden durch den Nutzer eingerichtet und bestehen aus vier bzw. sechts Ziffern oder einem alphanumerischen 
Code mit beliebiger Länge. Der Gerätecode wird nicht nur zum Entsperren des Geräts verwendet, sondern bestimmt 
die Entropie bestimmter Verschlüsselungscodes. Die Architektur verknüpft den Gerätecode mit der UID, einer eindeutigen
Geräte-ID, wodurch bspw. Brute-Force-Attacken nur direkt auf dem Endgerät durchgeführt werden können.
Diese Form der Attacke wird zudem, durch einen zunehmenden Timeout bei falschen Gerätecode-Eingaben, erschwert.
\\

\begin{table}[h]
	\centering
	\begin{tabular}{c|c}
	 Versuche & Erzwungene Verzögerung (Timeout) \\
	\hline
	 1 - 4 & Keine  \\
	\hline 
	 5 & 1 Minute  \\
	\hline 
	 6 & 5 Minuten  \\
	\hline 
	 7 - 8 & 15 Minuten  \\
	\hline 
	 9 &  1 Stunde  \\
	\end{tabular}
	  \caption{Timeouts bei falschen Gerätecode-Eingaben}
\end{table}


Apple gibt an, dass es über fünfeinhalb Jahre dauern würde, um alle sechsstelligen alphanumerischen Gerätecodes mit 
Kleinbuchstaben und Zahlen auszuprobieren \cite[S. 63]{apple2020}. Es liegt auf der Hand, dass ein starker Gerätecode 
(Länge und Komplexität) zu einer höheren Sicherheit führt. Das Betriebssystem bietet zudem die Option, 
bei zehn aufeinander folgenden Fehlversuchen alle Daten des Gerätes zu löschen. \\

Als Alternative zum klassischen Passwort hat Apple weitere Authentifizierungsverfahren wie Touch- und Face ID eingeführt. 
Die biometrischen Daten stellen einen deutlich stärkeren Gerätecode dar, der die Wirksamkeit der Entropie der 
Verschlüsselungsschlüssel faktisch erhöht und dadurch die Datensicherheit verbessert. Jedoch wurde in einer App von
Citrix Worx die Touch ID Authentifizierung durch ein definiertes Prozedere ausgehebelt, so dass es Angreifern möglich 
war, auf geschützte Bereiche dieser App zuzugreifen. Die Ursache lag in einer fehlerhaften Implementierung, die dafür 
gesorgt hat, dass der Gerätecode falsch gespeichert wurde (Improper platform usage, Mobile Top10 2016, OWASP). \\


\subsection{Sicherheit vor unautorisierten  Datenverbindungen}
Physisch angeschlossene Geräte,  z.B. Mac, PC oder Zubehör, werden über die Lightning-, USB- oder Smart 
Connector-Schnittstelle mit dem Apple-Gerät verbunden. Aufgrund der Tatsache, dass die Ökosysteme der
angeschlossenen Geräte vielfältig sein können und es keine kryptographisch zuverlässige Möglichkeit gibt,
die Geräte vor einer Datenverbindung zu identifizieren, verlangen iPhones bzw. iPads die Eingabe des Gerätecodes,
bevor eine Datenverbindung aufgebaut werden kann. \cite[S. 65]{apple2020} Dies verhindert den unautorisierten
Zugriff auf die Daten durch ein Fremdsystem. Nachdem die Datenverbindung zu einem Fremdsystem autorisiert wurde, 
muss der Gerätecode, für eine bestimmte Zeitspanne, nicht mehr für den Verbindungsaufbau zu dem System 
eingegeben werden. Dadurch können Nutzer, die regelmäßige Verbindungen zu bestimmten Geräten herstellen möchten,
für eine Zeitspanne auf eine erneute Authentifizierung verzichten. \\

\subsection{Schutz persönlicher Daten}
Mobile Anwendungen benötigen oftmals den Zugriff auf bestimmte Funktionaliten oder Daten des Smartphones.
Diese stehen den Anwendungen aber nicht einfach zur Verfügung, Nutzer können für jede App festlegen, auf 
welche personenbezogene Informationen zugegriffen werden darf. Realisiert wird dies mit verschiedenen Technologien 
wie bspw. dem Data Vault. In den Einstellungen des Betriebssystems kann der Nutzer einer App die Rechte für den Zugriff 
auf individuelle Informationen geben. In einigen systemrelevanten Apps von iOS und iPadOS wird der Zugriff jedoch erzwungen, 
dazu zählen bspw. Kalender, Kamera, Bluetooth und Fitness (Health). \\ 

Seit der iOS bzw. iPadOS Version 13.4 werden die Daten der Apps von externen Anbieter (nicht Apple) in einem Data Vault geschützt, 
um den Schutz vor unbefugten Datenzugriffen zu gewährleisten. Die Apps unter iOS und iPadOS erhalten bei der Nutzung von iCloud 
standardmäßig Zugriff auf die iCloud Drive. Der Nutzer kann den iCloud-Zugriff jedoch in den Einstellungen ändern und einschränken. 

\subsection{Klassifizierung der Sicherheitschipes}
Im Rahmen dieser Ausarbeitung werden oftmals die Sicherheitschips (SoCs - System on Chip) der mobilen Endgeräte thematisiert. 
Die meisten dieser Sicherheitschips basieren auf einer ARM-Architektur und werden in die meisten der Apple-Produkte wie bspw.
iPhone, iPad, Apple Watch, etc. verbaut. Die Chips unterscheiden sich dabei u.a. durch unterschiedliche Anwendungszwecke, 
Fertigungstechnologien, integrierte Co-Prozessoren, AI-Beschleuniger und der Anzahl der integrierten Transistoren. Für eine 
bessere Übersicht werden folgend die verschiedenen SoC-Serien, und die Produkte in denen sie verbaut werden, aufgelistet.

\begin{itemize}
\item \textit{A series}: iPhone und iPad
\item \textit{S series}: Apple Watch
\item \textit{T series} series: macOS-Produkte (MacBook, Mac-Computer)
\item \textit{W series}: AirPods und HomePod
\end{itemize}



