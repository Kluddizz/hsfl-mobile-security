\section{Zugriffskontrolle}
Innerhalb einer App werden Informationen auf unterschiedlichste Art und Weise
verwaltet. Beispielsweise sind Anwendungen, die keinerlei Daten ablegen bzw. auf
die Datenträger der Plattform speichern, ziemlich langweilig. Häufig möchte man
Konfigurationen oder Teile der Geschäftslogik irgendwo auf dem Gerät ablegen, um
auch nach Neustart der App die jeweiligen Datensätze vorliegen zu haben und
nicht zu verlieren.

Das Speichern von Informationen in Dateien ist also ein sehr beliebtes Mittel,
um Daten persistent abrufbar zu gestalten -- auch über die Terminierung einer
App hinaus. Nur leider ist es auch möglich, dass Apps sensible Daten in Dateien
ablegen.  Es muss daher einen Mechanismus geben, welcher den unautorisierten
Zugriff auf sensible Daten innerhalb einer Datei verbietet bzw.  einschränkt.
Vor allem soll der Zugriff von anderen (unbekannten) Apps kontrolliert werden.
Hierfür existieren auf Apple-Plattformen sogenannte
\textit{Datensicherheitsklassen}, die von Apps verwendet werden, um den Zugriff
auf die von der App erstellten Dateien zu kontrollieren \cite[S. 50]{apple2020}.

Neben dem Speichern von Informationen im Allgemeinen benötigt man weitere,
sicherheitsrelevante Steuermechanismen für das Ablegen von sensiblen Daten. Vor
allem bei sehr sensiblen Informationen, wie beispielsweise Pass\-wör\-ter,
kryptographische Schlüssel und Anmelde-Tokens, wünscht man eine sichere
Speicherung, damit nicht jeder Benutzer bzw. jede App Zugriff erhält. Hierfür
bieten die Apple-Plattformen Schutzmechanismen wie \textit{Access-Control-Lists}
(ACL), um Schlüsselbünde abzusichern \cite[S. 55]{apple2020}.

\subsection{Informationsflusskontrolle von Dateien}
Bevor der Kontrollmechanismus für das Anlegen und Verwalten von Dateien auf
Apple-Plattformen diskutiert wird, schauen wir uns ein Beispiel an. Stellen wir
uns vor, es sollen militärische Dokumente verwaltet werden. Sie sollen gelesen
und verändert werden können. Da es sich meist um Dokumente handelt, die
besondere Diskretion erfordern, sollte klar sein, dass einfache Soldaten nicht
dieselben Rechte zum Lesen und Bearbeiten von solchen Dokumenten besitzen wie
Generäle. Um den Informationsfluss zu kontrollieren, klassifizieren wir die
Dokumente, sodass nur Entitäten mit dem entsprechenden militärischen Rang Lese-
und Schreibrechte besitzen. Das Einteilen und fortlaufende Kontrollieren von
Informationen und deren Fluss durch Anwendungen wird durch eine
\textit{Informationsflusskontrolle}, englisch \textit{Information Flow Control
(IFC)}, beschrieben. Kurz zusammengefasst erstellen wir \textit{Gitter} aus
Klassen, die letztenendes den Informationsfluss durch die Anwendung beschreiben.

\subsection{Kontrolle von Schlüsselbundzugriffen}
