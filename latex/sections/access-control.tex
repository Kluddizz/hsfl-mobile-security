\section{Zugriffskontrolle}
Innerhalb einer App werden Informationen auf unterschiedlichste Art und Weise
verwaltet. Beispielsweise sind Anwendungen, die keinerlei Daten ablegen bzw. auf
die Datenträger der Plattform speichern, ziemlich langweilig. Häufig möchte man
Konfigurationen oder Teile der Geschäftslogik irgendwo auf dem Gerät ablegen, um
auch nach Neustart der App die jeweiligen Datensätze vorliegen zu haben und
nicht zu verlieren.

Das Speichern von Informationen in Dateien ist also ein sehr beliebtes Mittel,
um Daten persistent abrufbar zu gestalten -- auch über die Terminierung einer
App hinaus. Nur leider ist es auch möglich, dass Apps sensible Daten in Dateien
ablegen.  Es muss daher einen Mechanismus geben, welcher den unautorisierten
Zugriff auf sensible Daten innerhalb einer Datei verbietet bzw.  einschränkt.
Vor allem soll der Zugriff von anderen (unbekannten) Apps kontrolliert werden.
Hierfür existieren auf Apple-Plattformen sogenannte
\textit{Datensicherheitsklassen}, die von Apps verwendet werden, um den Zugriff
auf die von der App erstellten Dateien zu kontrollieren \cite[S. 50]{apple2020}.

Neben dem Speichern von Informationen im Allgemeinen benötigt man weitere,
sicherheitsrelevante Steuermechanismen für das Ablegen von sensiblen Daten. Vor
allem bei sehr sensiblen Informationen, wie beispielsweise Pass\-wör\-ter,
kryptographische Schlüssel und Anmelde-Tokens, wünscht man eine sichere
Speicherung, damit nicht jeder Benutzer bzw. jede App Zugriff erhält. Hierfür
bieten die Apple-Plattformen Schutzmechanismen wie \textit{Access-Control-Lists}
(ACL), um Schlüsselbunde abzusichern und den Zugriff auf Ihnen einzuschränken.
\cite[S. 55]{apple2020}.

\subsection{Informationsflusskontrolle von Dateien}
Bevor der Kontrollmechanismus für das Anlegen und Verwalten von Dateien auf
Apple-Plattformen diskutiert wird, schauen wir uns ein Beispiel an. Stellen wir
uns vor, es sollen militärische Dokumente verwaltet werden. Sie sollen gelesen
und verändert werden können. Da es sich meist um Dokumente handelt, die
besondere Diskretion erfordern, sollte klar sein, dass einfache Soldaten nicht
dieselben Rechte zum Lesen und Bearbeiten von solchen Dokumenten besitzen wie
Generäle. Um den Informationsfluss zu kontrollieren, klassifizieren wir die
Dokumente, sodass nur Entitäten mit dem entsprechenden militärischen Rang Lese-
und Schreibrechte besitzen. Das Einteilen und fortlaufende Kontrollieren von
Informationen und deren Fluss durch Anwendungen wird durch eine
\textit{Informationsflusskontrolle}, englisch \textit{Information Flow Control
(IFC)}, beschrieben. Kurz zusammengefasst erstellen wir \textit{Gitter} aus
Klassen, die letztenendes den Informationsfluss durch die Anwendung beschreiben
\cite{ifc}.

Beim nächsten Beispiel handelt es sich um eine App, die beispielsweise
Kontaktdaten in eine Datei auf das Gerät (iOS) speichert. Auch hier möchten wir
sicherstellen, dass die Informationen nur unter bestimmten Umständen ausgelesen
werden können.  Genau wie bei der Informationsflusskontrolle. Wenn eine App auf
iOS eine Datei schreibt, wird dieser Datei laut \cite{appledev,apple2020}
zusätzlich eine Datensicherheitsklasse zugewiesen, um dieses Ziel zu erreichen.
Eine Sicherheitsklasse gibt an, wann auf die Informationen zugegriffen bzw.
wann die Datei gelesen werden darf. Hierfür wurden vier Klassen implementiert,
die im Folgenden kurz erläutert werden.

\paragraph{Vollständiger Schutz.}{Die Verschlüsselung des Klassenschlüssels wird
mithilfe des Benutzercodes und der eindeutigen UID \textit{(unique identifier)}
des Geräts realisiert. Nachdem das Gerät gesperrt wird, wird der entschlüsselte
Klassenschlüssel verworfen und erst bei erneuter Entsperrung entschlüsselt.
Dateien mit dieser Sicherheitsklasse können also nicht gelesen oder geschrieben
werden, solange das Gerät gesperrt ist.}

\paragraph{Geschützt, außer wenn offen.}{Wenn diese Sicherheitsklasse für eine
Datei verwendet wird, kann diese erstellt werden, auch wenn das Gerät gesperrt
wird. Ist der Erstellungsvorgang abgeschlossen, kann diese Datei erst wieder
gelesen werden, wenn das Gerät entsperrt wird.}

\paragraph{Geschützt bis zur ersten Authentifizierung.}{Genau wie die Klasse
\textit{Vollständiger Schutz} wird mithilfe des Benutzercodes und der UID des
Geräts ein Klassenschlüssel entschlüsselt. Der Unterschied liegt nun darin, dass
dieser entschlüsselte Klassenschlüssel beim Sperren des Geräts nicht verworfen
wird. Erst bei Neustart muss dieser erneut entschlüsselt werden. Dieser Schutz
ist vergleichbar mit der Verschlüsselung der Festplatten bei Desktoprechnern.}

\paragraph{Kein Schutz.}{Der Klassenschlüssel wird hier nur mithilfe der UID des
Geräts entschlüsselt bzw. geschützt. Auch wenn einer Datei keine
Sicherheitsklasse zugewiesen wurde, wird diese vom Betriebssystem verschlüsselt
gespeichert.}

\subsection{Kontrolle von Schlüsselbundzugriffen}
Damit nicht jeder auf die Schlüsselbunde eines Geräts zugreifen kann, werden
\textit{Access-Control-Lists} (ACLs) verwendet, die Richtlinien für den Zugriff
und Authentifizierung bereitstellt. Das Ziel ist unter anderem, dass der
Benutzer des Geräts anwesend ist und sich mittels Touch ID oder anderen
Authentifizierungsmethoden ausweist. ACLs werden in der \textit{Secure-Enclave}
geprüft und erst dann an den Kernel weitergegeben, falls alle Bedingungen
erfüllt wurden \cite{apple2020}.

Dies allein ist jedoch nicht Schutz genug gegen einen Angreifer. Betrachten wir
folgendes Beispiel, wird die Problematik klarer. Falls das Gerät in die Hände
eines Angreifers gerät, wird dieser versuchen, Zugriff auf Schlüs\-sel\-bun\-de zu
bekommen. Hierfür legt sich der Angreifer einfach einen neuen Fingerabdruck an
und authentifiziert sich mit diesem. Falls nur die Anwesenheit eines
authentifizierten Benutzers geprüft werden würde, würde also nicht
sichergestellt werden, dass nur die bis zum Angriff bekannten Benutzer anwesend
sind. Hierfür kann zusätzlich kontrolliert werden, ob eine Touch ID verändert
bzw. wann sie erstellt wurde, um den Zugriff zu kontrollieren \cite{apple2020}.
