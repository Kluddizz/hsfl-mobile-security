\section{Gerätecodes}
Ein Gerätecode wird von dem Benutzer des Geräts eingegeben, um sicherzustellen,
dass dieser die erforderlichen Zugriffsrechte besitzt. Ist ein Gerätecode
eingerichtet, so wird gleichzeitig automatisch die Datensicherheit aktiviert.
Das bedeutet, dass die Dateien des Systems verschlüsselt vorliegen und je nach
Datensicherheitsklasse beim Entsperren des Geräts entschlüsselt werden. Ein
Gerätecode besteht aus vier oder sechs Ziffern, jedoch können auch
alphanumerische Zeichenfolgen (\texttt{a-z, A-Z, 0-9}) mit beliebiger Länge als
Gerätecode eingerichtet werden. Natürlich haben diese Codes dann eine höhere
Sicherheit als die Varianten, die nur aus Ziffern bestehen. Wichtig zu erwähnen
ist außerdem, dass die Gerätecodes mit der UID des Geräts verbunden sind. Eine
Authentifizierung ist deshalb nur auf dem Zielgerät möglich und stellt eine
Sicherheitsvorkehrung dar, da das Gerät im physischen Besitz des Angreifer sein
muss. Selbst wenn das der Fall ist, existiert ein Mechanismus, der den
Authentifizierungsprozess auf 80 Millisekunden festsetzt. Damit sind
Brute-Force-Angriffe schon bei vierstelligen Gerätecodes, bestehend aus Ziffern,
sehr unattraktiv für Angreifer. Zusätzlich können Touch ID oder Face ID als
Gerätecodes verwendet werden. Dies bietet dem Benutzer eine bessere
User-Experience und eine stärkere Sicherheit beim Verschlüsseln von Daten.
