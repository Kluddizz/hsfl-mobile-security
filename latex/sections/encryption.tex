\section{Verschlüsselung}
Kryptographische Schlüssel werden von dem Betriebsystem für die Ver- und Entschlüsselung von 
Informationen verwendet. Daten werden verschlüsselt, um die darin enthaltenden Informationen 
vor potentielle Angreifer geheimzuhalten. Die Verschlüsselung von Daten kann vielfältig eingesetzt werden, 
typische Anwendungszwecke gehen von der Verschlüsselung einzelner Dateien bis zum gesamten internen Dateisystem.


\subsection{Verschlüsselungalgorithmus}
Für die Verschlüsselung der Informationen zwischen Hardware-Komponenten oder
Applikationen werden unter iOS unterschiedliche Varianten des etablierten Verschlüsselungsverfahrens
AES (Advanced Encryption Standard) eingesetzt. AES ist ein symmetrisches Verschlüsselungsverfahren,
die miteinander interagierenden Parteien müssen also einen gemeinsamen Schlüssel austauschen, welcher
daraufhin für das ver- und entschlüsseln von Informationen verwendet wird. [Grafik]

\pagebreak

\subsection{Hardwaresicherheit}
Anhand der Beschreibung des symmetrischen Verschlüsselungsverfahren wird deutlich, dass die Schlüssel in
gesicherten Umgebung verwaltet und verwendet werden müssen. Anderenfalls kann nicht von der Vertraulichkeit 
der Daten ausgegangen werden, weil ein Angreifer möglicherweise durch den Diebstahl der Schlüssel den Datenverkehr
mitlesen kann. \\

Aus diesem Grund werden in modern iOS-, iPadOS- und watchOS-Geräten Sicherheitschips verbaut, die einen sicheren
Coprozessor enhalten. Dieser Coprozessor wird auch \textit{Secure Enclave} genannt und ist ein hardwarebasierter 
Schlüsselmanager, der von dem Hauptprozessor isoliert ist. Dadurch wird eine weitere Sicherheitsebene in die 
Verwaltung der sicherheitskritischen Schlüssel umgesetzt. Die Verschlüsselungsschlüssel werden nicht mal
der CPU oder dem Kernel des Betriebsystemes offengelegt, weil diese potenziell von einem Angreifer manipuliert
werden können. Außerdem enthält der Sicherheitschip eine Hardware-AES-Engine, die für die Ver- und Entschlüsselung
von Dateien mithilfe des AES-Verschlüsselungsalgorithmus eingesetzt wird. Während des Start des Systemes tauschen
die Secure Enclave und die AES-Engine einen temporären Schlüssel miteinander aus. Mithilfe des temporären Schlüssels
können die beiden Komponenten sicher innerhalb des Sicherheitschips kommunizieren. \\

\subsection{Schlüsselübersicht}
Das Betriebssystem verwendet verschiedene Schlüssel für unterschiedliche Funktionen.


* Hardwareverschlüsselung
* Klassenschlüssel
* Dateisystemschlüssel, Dateischlüssels, Zufallsschlüssel, Key-Wrapping-Schlüssels, Metadatenschlüssel
* Datensicherheitsschlüssel
* Medienschlüssel
