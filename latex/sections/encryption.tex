\section{Verschlüsselung}
Kryptographische Schlüssel werden von dem Betriebsystem für die Ver- und Entschlüsselung von 
Informationen verwendet. Daten werden verschlüsselt, um die darin enthaltenden Informationen 
vor potentielle Angreifer geheimzuhalten. Die Verschlüsselung von Daten kann vielfältig eingesetzt werden, 
typische Anwendungszwecke können bspw. die Verschlüsselung einzelner Dateien oder gar des gesamten internen Dateisystems sein.


\subsection{Verschlüsselungalgorithmus}
Für die Verschlüsselung der Datenaustausches zwischen Komponenten, Systemen oder Apps
werden unter iOS unterschiedliche Varianten des etablierten Verschlüsselungsverfahrens
AES (Advanced Encryption Standard) eingesetzt. AES ist ein symmetrisches Verschlüsselungsverfahren,
die miteinander interagierenden Parteien müssen also einen gemeinsamen Schlüssel austauschen, welcher
daraufhin für das ver- und entschlüsseln von Informationen verwendet wird. \\

	\begin{figure}[h]
		\centering
		\includegraphics[width=135mm]{images/symmetric.png}
		\caption{Symmetrische Verschlüsselung}
		\label{fig:symmetric}
	\end{figure}

\pagebreak

\subsection{Hardwaresicherheit}
Anhand der Beschreibung des symmetrischen Verschlüsselungsverfahren wird deutlich, dass die Schlüssel in
gesicherten Umgebung verwaltet und verwendet werden müssen. Anderenfalls kann nicht von der Vertraulichkeit 
der Daten ausgegangen werden, weil ein Angreifer möglicherweise durch den Diebstahl der Schlüssel den Datenverkehr
mitlesen kann. \\

Aus diesem Grund werden in modern iOS-, iPadOS- und watchOS-Geräten Sicherheitschips verbaut, die einen sicheren
Coprozessor enhalten. Dieser Coprozessor wird auch \textit{Secure Enclave} genannt und ist ein hardwarebasierter 
Schlüsselmanager, der von dem Hauptprozessor isoliert ist. Dadurch wird eine weitere Sicherheitsebene in die 
Verwaltung der sicherheitskritischen Schlüssel umgesetzt. Die Verschlüsselungsschlüssel werden nicht mal
der CPU oder dem Kernel des Betriebsystemes offengelegt, weil diese potenziell von einem Angreifer manipuliert
werden können. Außerdem enthält der Sicherheitschip eine Hardware-AES-Engine, die für die Ver- und Entschlüsselung
von Dateien mithilfe des AES-Verschlüsselungsalgorithmus eingesetzt wird. Während des Start des Systemes tauschen
die Secure Enclave und die AES-Engine einen temporären Schlüssel miteinander aus. Mithilfe des temporären Schlüssels
können die beiden Komponenten sicher innerhalb des Sicherheitschips kommunizieren. \\

\subsection{Schlüssel- und Passwörterverwaltung}
Sicherheitskritische Daten wie kryptographische Schlüssel oder Passwörter müssen in einer gesicherten Umgebung
verwaltet werden können. Mit der Secure Enclave wurde bereits eine Methode zum Speichern dieser Daten erläutert, 
in den folgenden Abschnitten werden weitere Methoden, die iOS und iPadOS verwenden, vorgestellt.

\subsubsection{Schlüsselbund}
Mobile Anwendungen müssen oftmals vertrauliche Daten, wie bspw. Passwörter oder Tokens, auf dem Gerät hinterlegen. 
Das Betriebssystem bietet dafür den sogenannten Schlüsselbund an. Diese Komponente ist eine SQLite-Datenbank,
welche die sensiblen Daten in verschlüsselter Form speichert. Der Schlüsselbund befindet sich dabei \textbf{nicht} in der
Secure Enclave, sondern im Gerätespeicher. Dies ist aber dennoch sicher, weil der Verschlüsselungsschlüssel, welcher
für die Entschlüsselung der Daten benötigt wird, sich in der Secure Enclave befindet.

\subsubsection{Keybags}
Keybags sammeln und verwalten die Schlüssel der verschiedenen Datensicherheitsklassen. Jeder Keybag verwaltet aber
nur die Klassenschlüssel seines Types, im Kapitel \textit{X. Information Flow Control} werden diese vorgestellt und ihre 
Funktionsweise erklärt.


